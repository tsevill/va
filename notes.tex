\chapter{Notes}
ASSESSING AGRICULTURAL VULNERABILITY TO CLIMATE CHANGE IN THE NORDIC COUNTRIES – AN INTERACTIVE GEOVISUALIZATION APPROACH: ~\cite{wirehn2016assessing}\\
\textit{Nordic agriculture must adapt to climate change to reduce vulnerability and exploit potential opportunities. Integrated assessments can identify and quantify vulnerability in order to recognize these adaptation needs. This study presents a geographic visualization approach to support the interactive assessment of agricultural vulnerability to climate change. We have identified requirements for increased transparency and reflexivity in vulnerability assessments, arguing that these can be met by geographic visualization. A conceptual framework to support the integration of geographic visualization for vulnerability assessments has been designed and applied for the development of AgroExplore, an interactive tool for assessing agricultural vulnerability to climate change in Sweden. To open up the black box of composite vulnerability indices, AgroExplore enables the user to select, weight, and classify relevant indicators into sub-indices of exposure, sensitivity, and adaptive capacity. This enables the exploration of underlying indicators and factors determining vulnerability in Nordic agriculture.}\\
Defines potential similar workflows for how manipulation of the indicators would propogate into vulnerability indices. Social indicators may be important as well. Number of farmers and other employment in agriculture. Where can we get statistics for these? Extension of this agricultural vulnerability tool into the economic and trade network realms. What is the direct connection between agricultural vulnerability and trade network stability or vulnerability. There a lots of potential references here.\\
\\
ASSESSMENT OF COMPOSITE INDEX METHODS FOR AGRICULTURAL VULNERABILITY TO CLIMATE CHANGE:~\cite{wirehn2015assessment}\\
\textit{A common way of quantifying and communicating climate vulnerability is to calculate composite indices from indicators, visualizing these as maps. Inherent methodological uncertainties in vulnerability assessments, however, require greater attention. This study examines Swedish agricultural vulnerability to climate change, the aim being to review various indicator approaches for assessing agricultural vulnerability to climate change and to evaluate differences in climate vulnerability depending on the weighting and summarizing methods. The reviewed methods are evaluated by being tested at the municipal level. Three weighting and summarizing methods, representative of climate vulnerability indices in general, are analysed. The results indicate that 34 of 36 method combinations differ significantly from  each other. We argue that representing agricultural vulnerability in a single composite index might be insufficient to guide climate adaptation. We emphasize the need for further research into how to measure and visualize agricultural vulnerability and into how to communicate uncertainties in both data and methods.}\\
There are vulnerability indicators with descriptions here that may be useful. Groups of vulnerability indices: exposure, sensitivity, adaptive capacity. These groups will be possibly different for different regions. At least the index value will be different for different region. The indicators in each group will stay the same, but we can use it as a means of grouping the variables for a starting point. What I mean is possibly sliders that change the composite index of each group, instead of as a whole or as the individual indices. Adaptive capacity should remain static (or at least independent of climate singularities) in future predictions, but could be manipulated for exploration. Good indexing of the correlation between indicator and whether it has a positive of negative impacted on the overall vulnerability index.\\
\\
INDICATORS OF VULNERABILITY AND ADAPTIVE CAPACITY: ~\cite{hinkel2011indicators}\\
\textit{The issue of 'measuring' climate change vulnerability and adaptive capacity by means of indicators divides policy and academic communities. While policy increasingly demands such indicators an increasing body of literature criticises them. This misfit results from a twofold confusion. First, there is confusion about what vulnerability indicators are and which arguments are available for building them. Second, there is confusion about the kinds of policy problems to be solved by means of indicators. This paper addresses both sources of confusion. It first develops a rigorous conceptual framework for vulnerability indicators and applies it to review the scientific arguments available for building climate  change vulnerability indicators. Then, it opposes this availability with the following six diverse types of problems that vulnerability indicators are meant to address according to the literature: (i) identification of mitigation targets; (ii) identification of vulnerable people, communities, regions, etc.; (iii) raising awareness; (iv) allocation of adaptation funds; (v) monitoring of adaptation policy; and (vi) conducting scientific research. It is found that vulnerability indicators are only appropriate for addressing the second type of problembut only at local scales, when systems can be narrowly defined and inductive arguments can be built. For the other five types of problems, either vulnerability is not the adequate concept or vulnerability indicators are not the adequate methodology. I conclude that both the policy and academic communities should collaboratively attempt to use a more specific terminology for speaking about the problems addressed and the methodologies applied. The one-size-fits-all vulnerability label is not sufficient. Speaking of 'measuring' vulnerability is particularly misleading, as this is impossible and raises false expectations.}\\
Vulnerability is a measure of possible future harm. What are measures in the realm of trade networks. Are we talking merely profit or the overall economic well being of a region. There is not a real good scientific definition that provides guidance for designing a methodology. This paper will be useful in structuring our definition of vulnerability, as it seems to be a common theme that it is difficult to define. Reread the challenges involved in developing indicators. Distinguishing between harm and vulnerability. Harm indicators are a current state; good or bad state based on normative judgements that do not include the forward-looking aspect. Vulnerability makes the distinction of looking forward aspect as well as defining harm. Using normative arguments in the development of indicators means using (individual or collective) value judgements. Paper poses a large number of questions to ask when consider whether an indicator should be presented. Could use it to validate our indicator selections.\\
\\
INTEGRATING SOCIAL VULNERABILITY INTO WATER MANAGEMENT: ~\cite{downing2006integrating}\\
\textit{The vulnerability of humans related to their use of water has also become a widespread concern, related to climate change, flood and drought hazards, and poverty. A profusion of definitions remains characteristic of vulnerability research and applications. Nevertheless, progression in the past decade toward a vulnerability/adaptation science has recognised six key attributes of social vulnerability. Each implies different methodological approaches:
\begin{enumerate} \item Vulnerability is the differential exposure to stresses experienced or anticipated by different exposure units. \item Vulnerability is a dynamic process, changing on a variety of inter-linked time scales.
	\item Social vulnerability is rooted in the actions and multiple attributes of human actors.
	\item Social networks drive and bound vulnerability in the social, economic, political and
	environmental interactions.
	\item Vulnerability is constructed simultaneously on more than one scale.
	\item Multiple stresses are inherent in integrated vulnerability of peoples, places and systems.
\end{enumerate}
Building upon a typical water planning approach (such as WEAP), four progressions are proposed in understanding vulnerability: (1) introduce differential social and economic vulnerability to catchment planning models; (2) capture the dynamic element of vulnerable groups and their relationship to water resources and catchment or regional planning; (3) represent the multiple attributes of vulnerable groups and make the link to their ability to respond to stresses and threats; and (4) represent the decisions of actors (the managers and vulnerable groups) in the construction of adaptive systems (i.e., in the reduction of future vulnerability). The basic elements of a variety of water resource vulnerability recipes are reviewed.}\\
Notes\\
\\
NODETRIX: ~\cite{henry2007nodetrix}\\
\textit{The need to visualize large social networks is growing as hardware capabilities make analyzing large networks feasible and many new data sets become available. Unfortunately, the visualizations in existing systems do not satisfactorily resolve the basic dilemma of being readable both for the global structure of the network and also for detailed analysis of local communities. To address this problem, we present NodeTrix, a hybrid representation for networks that combines the advantages of two traditional representations: node-link diagrams are used to show the global structure of a network, while arbitrary portions of the network can be shown as adjacency matrices to better support the analysis of communities. A key contribution is a set of interaction techniques. These allow analysts to create a NodeTrix visualization by dragging selections to and from node-link and matrix forms, and to flexibly manipulate the NodeTrix representation to explore the dataset and create meaningful summary visualizations of their findings. Finally, we present a case study applying NodeTrix to the analysis of the InfoVis 2004 coauthorship dataset to illustrate the capabilities of NodeTrix as both an exploration tool and an effective means of communicating results.}\\
Technique was to split the visualization based on the two different densities of data at different spectrums. Locally the information was dense and therefore warranted a matrix representation. The community connections however were sparse and were better represented with a node-link diagram. This novel approach combined the two visualizations and allowed for user manipulation to condense the nodes into matrices.
What I garnered from this paper was the different types of clusters and what visualizations lend best represent each density. The term “small-world networks” was introduced to me as an intermediate category of data that has both sparse and dense clusters. These are common in data from social networks. Adjacent matrices are well suited for dense data such as our data set of exports from a county. However, there may be many countries with many different values that may not lend itself well to such a simplistic view model. We need to be able to represent quantity in a better fashion, and that’s why flow maps would be better suited than matrices. The paper says “matrix representations ease community analysis, but hinder identification of important global structures” which is most definitely pertinent to our data.
There was a lot of useful information on editing and information regarding user interface overall. The manipulation of matrices and nodes seemed intuitive and as I was reading I found myself having ideas that they presented shortly after.\\
\\
REDUCING SNAPSHOTS TO POINTS: ~\cite{van2016reducing}\\
\textit{We propose a visual analytics approach for the exploration and analysis of dynamic networks. We consider snapshots of the network as points in high-dimensional space and project these to two dimensions for visualization and interaction using two juxtaposed views: one for showing a snapshot and one for showing the evolution of the network. With this approach users are enabled to detect stable states, recurring states, outlier topologies, and gain knowledge about the transitions between states and the network evolution in general. The components of our approach are discretization, vectorization and normalization, dimensionality reduction, and visualization and interaction, which are discussed in detail. The effectiveness of the approach is shown by applying it to artificial and real-world dynamic networks.}\\
Network evolution is important if we are going to be visualizing current and potential futures states. There are two different basic approaches: time to time and time to space. A key point in creating an effective representation is a balance between fewer images lacking temporal data and many images lacking interpretability. There are two dominant methods for these types of data. Animation, where you see time lapsed by changing visualizations and small multiples where there are point in time visualizations based on a time interval. One the important considerations is mental map preservation. Being able to adjust the quantity and visualizations of data to be able to be effectively retained, either in a session state or more long term. Initially I think animation would be the most appropriate method for visualizing a flow map over a time period. The idea to break up the approach into these 4 steps could be translated to a flow map, this would still need to determined and explored further.\\\textsl{}
\\
TOWARDS A FORMAL FRAMEWORK OF VULNERABILITY TO CLIMATE CHANGE: ~\cite{ionescu2009towards}\\
\textit{There is confusion regarding the notion of “vulnerability” in the climate change scientific community. Recent research has identified a need for formalisation, which would support accurate communication and the elimination of misunderstandings that result from ambiguous interpretations. Moreover, a formal framework of vulnerability is a prerequisite for computational approaches to its assessment. This paper presents an attempt at developing such a formal framework. We see “vulnerability” as a relative concept, in the sense that accurate statements about vulnerability are possible only if one clearly specifies (i) the entity that is vulnerable, (ii) the stimulus to which it is vulnerable and (iii) the preference criteria to evaluate the outcome of the interaction between the entity and the stimulus. We relate the resulting framework to the IPCC conceptualisation of vulnerability and two recent vulnerability studies.}\\
\\
VULNERABILITY: ~\cite{adger2006vulnerability}\\
\textit{This paper reviews research traditions of vulnerability to environmental change and the challenges for present vulnerability research in integrating with the domains of resilience and adaptation. Vulnerability is the state of susceptibility to harm from exposure to stresses associated with environmental and social change and from the absence of capacity to adapt. Antecedent traditions include theories of vulnerability as entitlement failure and theories of hazard. Each of these areas has contributed to present formulations of vulnerability to environmental change as a characteristic of social-ecological systems linked to resilience. Research on vulnerability to the impacts of climate change spans all the antecedent and successor traditions. The challenges for vulnerability research are to develop robust and credible measures, to incorporate diverse methods that include perceptions of risk and vulnerability, and to incorporate governance research on the mechanisms that mediate vulnerability and promote adaptive action and resilience. These challenges are common to the domains of vulnerability, adaptation and resilience and form common ground for consilience and integration.}\\
\\
AGRICULTURAL TRADE NETWORK AND PATTERNS OF ECONOMIC DEVELOPMENT: ~\cite{shutters2012agricultural}\\
There are similarities of agricultural trade network (ATN) with known triad significance profile (TSP) of human social networks; and biological information processing networks to a lesser degree.  Is there a separate superfamily of TSPs for ATNs? Learned what in-degree and out-degree definitions of nodes are. (1, 0) indicates 1 incoming edge (imports) and 0 outgoing edges (exports).
Triads were the main take away from this paper. There are direct connections that can be made with my thesis. Some thinking points that came up were how we would effectively visualize the triads. Is this is an effective method for how we want to display the related information? 
There are special cases of the 18 ‘isolated’ countries when it comes to trade networks. Do we need to factor this in when visualizing our information?  Should the prediction algorithms and the filtering take these into account? Are we able to predict with changes to virtual water supply or climate change singularities that a country would enter or exit this categorization? Can we determine what steps a country would need to take to get themselves out of this isolation group?
It would be good to review the work done on the global virtual water trade network reference 10 in this paper.\\
\\
A NETWORK APPROACH TO DIVISION OF LABOR IN ANTS: ~\cite{networkApproach}\\
This article provided more information and background on triads and in/out-degrees. These associations of nodes, in this case ants giving or receiving signals, provide a different type of possible categorization; specializations. Are there specialized countries with respect to the ATN? Are there countries that are primarily exporters or primarily importers? Is this an indicator of economic status/development as in the ‘Agricultural Trade Network and Patterns of Economic Development’ article? Which triads with respect to ATN are indicators of prominence? How many categories of specialization are there? Which triads are indicative of categorization?
Use this paper as reference for specialization based on triad involvement and more importantly which role (in/out degree) the node is. How many unique nodes that are contained within the triad are deterministic of the role a node takes? For example, is triad 1, topmost node (0, 2) [note, this is incorrectly labeled in the draft paper as (2, 0)] ‘major exporter’ because they are more developed? How does this breakdown based down on virtual water values for different products in the ATN.\\
\\
MOTIF SIMPLIFICATION: ~\cite{dunne2013motif}\\
Due to the large number, as we are seeing with creation of the histograms, of trade links with small values it may be useful to simplify some of clusters if it makes sense visually; e.g. the United States exports to each of the Caribbean islands, it we could show one export link in order to make it less cluttered. There is pseudo-code that may prove useful if this approach is implemented.
There are 3 main types of simplification introduced; fan, connector and clique. The above example would be considered a fan, where one node has edges to each of the others, but not each other. This is done at a binary level, but could be modified to use weighted values. For instance, there could possibly be network links back to the United States in the above example, but if the values (however we determine that; quantity or dollar value) are negligible compared to the main export link we could essentially ignore the return links.
This cluster could also prove useful in defining triads for easier comparison and visualization if the simplifications are relevant enough.\\
\\
INTERNATIONAL TRADE AND FINANCIAL INTEGRATION: ~\cite{schiavo2010international}\\
Three binary data points for nodes are defined; node degree (the number of links maintained by each node), average nearest neighbor degree (the correlation between the degree of a node and the average degree of its partners) and clustering coefficient (the percentage of pairs of i's partners that are connected among themselves). These three node data points are also reciprocated in a weighted fashion based on the relative node strength. In our case this would be the value (quantity or dollar value) of an export link.
This paper uses both import and export values for the edges, creating an undirected graph. This may not be desirable in our analysis. However there is data that supports that the degree of symmetry is large enough to make directed analysis unnecessary. This is the case in both binary and weighted graphs. We need to determine if this is true for specific goods, or am I just misunderstanding their statement. It doesn’t seem to make sense that the US exporting to Nigeria has symmetry with imports from Nigeria. The correlation between node degree and node strength is negative, suggesting that nodes with a small number of links tend to connect to hubs, i.e. to nodes with many partners and suggests that the international trade network is organized as a star-shaped network.\\
\\
CHALLENGING PROBLEMS OF GEOSPATIAL VISUAL ANALYTICS: ~\cite{andrienko2011challenging}\\
Analytic reasoning can be done at two different points; primarily by human analysts or primarily computational. This lends itself to two different types visual interactions; one where the pieces are manipulated by the user and on where the computational methods are manipulated by the user.  The second approach is what I had in mind from the inception of this research. Applications of different kinds of forecasting and prediction algorithms are what would be manipulated by users, not the actual predicted future data. E.g. using X forecasting model would predict N mm of rain in 2020 is manipulating the computational model as opposed to stating there will be M mm of rain in 2020. There are also the techniques of clustering as in Motif Simplification paper, projection, aggregation, pattern extraction and simulation. Simulation is on the techniques I think will be relevant to our project.
Geographical data lends itself well to this two dimensional representation as there is the natural mapping. The quality of data is ever increasing with respect to location. We are going to be utilizing space and time and so the VisMaster findings noted in the paper are pertinent. There is also the interesting application of a countermeasure to the temporal problem. In our application we can also take this into consideration; what if countries started retaining water in an attempt to mitigate ‘drought’ years or the like. Need to bank on the two capabilities of humans; employ previous knowledge (including common sense) and establish associations. The second one we would attempt to enhance.\\
\\
CRITICAL MATERIALS, U.S. IMPORT DEPENDENCE AND RECOMMENDED ACTIONS: ~\cite{silberglitt2015critical}\\
This testimony deals with the pseudo-monopoly of China when it comes to export of certain critical materials and the potential derogatory effects on the U.S. industrial sector. Is there a similar superpower in the agricultural world trade network? Can this be examined based on the number or value of trade links? Using the histograms we created can are we able to answer this question? What can we do to visualize this?
This also introduces some questions that we may need to answer for the ATN. How to recognize a developing pattern, two-tier pricing (see below; initial importer vs second-tier), price spikes or volatility (specifically can we expect volatility with climate singularities?) Are there secondary products that are similar to processed v. unprocessed ore or minerals? How can we identify those?\\
\\
EVOLUTIONARY CONSERVATION OF SPECIES' ROLES IN FOOD WEBS: ~\cite{stouffer2012evolutionary}\\
This contains information on the different unique motif positions. We will use this as a reference when identifying the positions, as well as the triad numerical classification. There is information contribution of a motif and frequency of a motif position to consider when factor the ‘value’ of a position. For instance, in the triad 1, is position 3 more valuable than position 2? Is it assumed that some of the imports to 2 from 3 are then exported to 1, and if so how much? What about for two-way trade links? Can we and if so how do we get this from the data we have? Does this affect the value? Do we need to consider any value adjustment on this second-tier export? Again there is the question of clustering and if there is any value in clustering.\\
\\