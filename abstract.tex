\begin{abstract}
	\label{abstractChapter}
	The rise in globalization has led to regional climate events having an increased effect on global food security. These indirect first- and second-order effects are generally geographically disparate from the region experiencing the climate event. Without understanding the topology of the food trade network, international aid may be naively directed to the countries directly experiencing the climate event and not to countries that will face potential food insecurity due to that event. This thesis focuses on the development of a visual analytics system for exploring second-order effects of climate change under the lens of global trade. In order to visualize how climate change impacts the world trade network of agricultural goods I have developed an interactive data visualization platform for analysis of the interaction between climate events and the trade network. The proposed visual analytics system focuses on visualizing current trade dependencies at a more granular level than the currently available tools and to aid in the identification of future vulnerabilities. To demonstrate the applicability of the tool, two case studies are described. The first case study focuses on the Chinese drought of 2011 and its impact on the global trade network and food security. The second case study will model the potential impact of a climate event affecting production in the United States, a large supplier of corn, to demonstrate the potential consequence of cascading effects in the global trade network.
\end{abstract}